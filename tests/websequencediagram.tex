% Another simple example of how to use the module.
% Google charts uses & as a separtor. This shows how to define a transformation
% that uses & as a separator
\usemodule[webfilter]

\startluacode
  userdata = userdata or {}
  userdata.webfilter = userdata.webfilter or {}

  local http  = require("socket.http")
  local ltn12 = require("ltn12")
  local url   = require("socket.url")

  function userdata.webfilter.websequence(name) 
    local content   = buffers.getcontent(name)
    local style    = "modern-blue" 
    local body     = "style=" .. style .. "&message=" .. url.escape(content)
    local response = {}
    local status, message = http.request {
      method = 'POST',
      url    = "http://www.websequencediagrams.com",
      source = ltn12.source.string(body),
      sink   = ltn12.sink.table(response),
      headers = {
        ["Content-Length"] = string.len(body),
      },
    }

    -- ConTeXt does not have a JSON parser. So, we do a quick and 
    -- dirty parsing.
    local quot  = lpeg.P('"') 
    local other = 1 - quot

    local img = other^0 * quot * lpeg.Cs(other^0) * quot * other 

    local location = lpeg.match(img,response[1])
    return location
  end
\stopluacode

\defineexternalfigure[websequence][width=\textwidth,maxheight=\textheight,method=png]

\definewebfilter
  [websequence]
  [prefix={http://www.websequencediagrams.com/},
   transform=userdata.webfilter.websequence,
   figuresetup=websequence]

\enabletrackers[third.externalfilter]
\enabletrackers[resolvers.schemes]

\starttext
  \startwebsequence
    Card Holder->ATM: select withdrawal options
    Card Holder->ATM: Specify amount
    ATM->Bank: Check availability of funds
    ATM->ATM: Eject card
    ATM->Card Holder: Prompt card holder to take card
    Card Holder->ATM: Take card
    ATM->ATM:Dispense cash
    ATM->Card Holder: Prompt card holder to take cash
    Card Holder->ATM: Take Cash
    ATM->Bank: Debit Account
    ATM->ATM: Wait for next customer
  \stopwebsequence

\stoptext
