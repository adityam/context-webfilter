% Another simple example of how to use the module.
% Google charts uses & as a separtor. This shows how to define a transformation
% that uses & as a separator
\usemodule[webfilter]

\startluacode
  userdata = userdata or {}
  userdata.webfilter = userdata.webfilter or {}
  function userdata.webfilter.google(name) 
    return thirddata.webfilter.transform.default(name, "&")
  end
\stopluacode

\defineexternalfigure[GoogleChart][width=\textwidth,maxheight=\textheight,method=png,frame=on]

\definewebfilter
  [GoogleCharts]
  [prefix={http://chart.apis.google.com/chart?},
   suffix={&chof=png},
   figuresetup=GoogleChart,
   transform=userdata.webfilter.google]

\enabletrackers[third.externalfilter]
\enabletrackers[resolvers.schemes]

\starttext
\startGoogleCharts
  cht=bhg
  chs=700x300
  chd=t:100,50,115,80
  chxt=x,y
  chxl=1:|Python|Java|Ruby|.NET
  chxr=0,0,120
  chds=0,120
  chco=4D89F9
  chbh=35,0,15
  chg=8.33,0,5,5
\stopGoogleCharts

\stoptext
